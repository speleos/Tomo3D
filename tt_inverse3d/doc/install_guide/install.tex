\documentclass[twoside,final,onecolumn]{article}

\usepackage{framed}
\usepackage{epsfig}
\usepackage{natbib}
\usepackage{graphicx}
\usepackage{amsmath}
\usepackage{hyperref}
\usepackage[a4paper, total={6in, 8in}]{geometry}
\usepackage{enumerate}
\usepackage{ragged2e}
\usepackage{parskip}
\usepackage{setspace}

\textheight = 220.0 true mm
\textwidth = 180.0 true mm
\setlength{\parindent}{0pt}
\setlength{\evensidemargin} { -8 true mm}
\setlength{\oddsidemargin }{ -8 true mm}

\newcommand{\ttinverse}{{\tt tt\_inverse3d}}
\newcommand{\intel}{Intel\textregistered}
\newcommand{\MPI}{\textbf{MPI}}
\newcommand{\Boost}{\textbf{Boost}}
\newcommand{\CXX}{C\nolinebreak\hspace{-.05em}\raisebox{.4ex}{\tiny\bf +}\nolinebreak\hspace{-.10em}\raisebox{.4ex}{\tiny\bf +}}
\newcommand{\gxx}{g\nolinebreak\hspace{-.05em}\raisebox{.4ex}{\tiny\bf +}\nolinebreak\hspace{-.10em}\raisebox{.4ex}{\tiny\bf +}}
\newcommand{\cmake}{\textbf{CMake}}

\begin{document}

\title{
TOMO3D package: tt\_inverse3d installation Guide - Version 0.1.0\\
} 

\author{Adri\`{a} Mel\'{e}ndez, Alain Miniussi \& Sergio Costa\\
  \href{mailto:melendez@icm.csic.es}{melendez@icm.csic.es},\\
  \href{mailto:adria.melendez.catalan@gmail.com}{adria.melendez.catalan@gmail.com},\\
  \href{mailto:alain.miniussi@oca.eu}{alain.miniussi@oca.eu}, \\
  \href{mailto:sergiocosta@icm.csic.es}{sergiocosta@icm.csic.es} \\
}
\maketitle

\tableofcontents

\section{Overview}

This guide describes the installation procedure for all the present contents in the TOMO3D package.
Please address any questions on this matter to the first author with copy for the other two authors.

\subsection{Current Status}

Presently this package contains the necessary files for the installation of \textbf{tt\_inverse3d},
the anisotropic 3-D joint refraction and reflection traveltime tomography parallel code. The rest of applications 
can be installed with their respective Makefile.\\

\subsection{Terms and Conditions}
The contents of TOMO3D package are free to use for academic purposes only. Industrial use is forbidden unless otherwise stated.
The authors are listed immediately after the title of this document and are not liable for any misuse or misfunction of the contents of this package.\\
Please let us know at \href{mailto:melendez@icm.csic.es}{melendez@icm.csic.es} that you have downloaded a copy of the package and specify if you wish
to receive update notifications via e-mail. In the future, we may consider automatic notifications.\\
When presenting your results please cite: \href{http://www.dx.doi.org/10.1093/gji/ggv292}{Mel\'{e}ndez, A., Korenaga, J., Sallar\`{e}s, V., Miniussi, A. \& Ranero, C.R. (2015).
TOMO3D: 3-D joint refraction and reflection traveltime tomography parallel code for active-source seismic data - synthetic test, \textit{Geophys. J. Int.}, 203, 158-174}.

\section{Installing tt\_inverse3d}

tt\_inverse3d uses some external tools and libraries. With a few exception, we adopt the most recent stable versions. The code is accelerated through hybrid parallelization: \MPI\footnote{{\textbf{M}essage \textbf{P}assing \textbf{I}nterface}: \href{https://computing.llnl.gov/tutorials/mpi} {Tutorial.}}
is used to dispatch the work among compute nodes and multithreading is used to distribute the tasks inside a node.
As a result, the end user can run the code on a variety of machines ranging from laptops (mostly for testing and developement) to huge HPC\footnote{{\bf H}igh {\bf P}erformance {\bf C}omputing.\\} facilities.
\newline
\newline
Depending on your institution and computing environment, the required tools and libraries may or may not be readily available.
You may have to install them yourself or ask the IT department. 
Also, depending on your performance requirements and expectations, you might want to use specific versions of the requested libraries and tools, assuming this is allowed by your institution.
\newline
Here we present the most generic installation procedure possible, trying to address some specific issues that have been known to be problematic. 
Note that each of the dependencies needed has its own installation documentation, and we recommend that you refer to it for any queries.

\subsection{Build System}

We use \href{http://www.cmake.org/}{cmake}, version $2.8$ or newer is required. If not already available on your platform, it can be easily installed from \href{http://www.cmake.org/install/}{http://www.cmake.org/install/}.

\subsection{Compilers}

The code is written in \CXX. Tested compilers include \href{https://gcc.gnu.org/}{\gxx} and \href{https://software.intel.com/en-us/c-compilers}{\intel\ compiler}. It is expected to work with \href{http://clang.llvm.org/}{clang} although it has not been tested.
\newline
\newline
If you are not using your platform default compilers ({\bf gcc/\gxx} on most Unix systems) make sure that the {\bf CC} and {\bf CXX} variables are properly set. For example, if you want to use the \intel\ compiler family, you should have:
\begin{framed}
\begin{verbatim}
$ echo "C++ compiler: $CXX, C compiler:  $CC"
C++ compiler: icpc, C compiler:  icc
$ 
\end{verbatim}
\end{framed}

\paragraph{Troubleshooting:}
The code relies on \href{http://www.stroustrup.com/C++11FAQ.html}{\CXX$11$} features, and
\cmake\ will try to find out the necessary compiler option to enable them. If any, it will test it for some \CXX $11$ feature.
If your compiler specific option is not detected, you can pass it through \cmake:
\begin{framed}
\begin{verbatim}
$ cmake -DCMAKE_CXX_FLAGS="<my specific options>" [...]
...
$
\end{verbatim}
\end{framed}

\subsection{MPI and Multithreading}

We use \MPI\ to distribute the computation over the network. Note that you will need an MPI implementation even if you compute on a single node.
Any decent MPI implementation should do. Tested implementations include \href{http://www.open-mpi.org}{OpenMPI} (version 1.8.4) and \href{https://software.intel.com/en-us/intel-mpi-library}{\intel \ MPI}.
\newline
\newline
Each MPI process dispatches part of its work among multiple threads. As of now, only the main threads perform MPI calls,
meaning that MPI runtime only needs to ensure the threading support tagged as \href{http://www.mpi-forum.org/docs/mpi-2.2/mpi22-report/node260.htm}{\textbf{MPI\_THREAD\_FUNNELED}}.
If such a support is not provided, a warning message will be printed and the application will hope for the best.
\newline
\newline
Some implementations of OpenMPI require specific configuration flags that are not necessarily compatible with all high-performance interconnect systems.
It is worth noticing that ignoring this issue is a widespread habit in the HPC community.

\subsection{Boost}

tt\_inverse3d uses various components of the \href{http://www.boost.org}{\Boost} library. Since those components include the Boost MPI interface,
which depends on your MPI distribution, it is very unlikely that your platform official Boost distribution will do. 
Depending on your situation, there are a few things you might want to check out:

\subsubsection{Boost was installed by the administrator}

You need to check which MPI implementation was used and be consistent with that choice. Also, you might need to take a look at \texttt{$<$boost root$>$/include/boost/mpi/config.hpp} to see if the \textbf{BOOST\_MPI\_HOMOGENEOUS} variable is uncommented,
assuming that you are running the code on homogeneous nodes regarding data layout. This is the default starting from Boost version $1.59.0$.

\subsubsection{Installing Boost on top of \intel\ MPI}

If you are using \intel\ MPI you should use the $1.59$ version of Boost. Previous versions might trigger bugs depending on your MPI tunning.
You can download the most recent version available from \url{http://www.boost.org/users/download/}.
The configuration of {\bf Boost.MPI} on top of \intel\ MPI is described in its documentation, but we provide a quick reminder.
First, go to the root of your Boost source distribution and run:
\begin{framed}
\begin{verbatim}
$./bootstrap.sh --prefix <boostdist> --with-toolset=intel-linux
...
$
\end{verbatim}
\end{framed}
where {\verb+<boostdist>+} is the directory where you want to install Boost.
\newline
\newline
This will create a file named {\tt project-config.jam}. We are going to edit that file to incorporate the MPI configuration.
First, we need to find out which include directories and which libraries to add when using \intel\ MPI.
Usually, the Boost build system is able to deduce this information through {\tt mpicc}'s {\tt -show:compile} and {\tt -show:link}.
Unfortunately, \intel\ implementation does not provide these options anymore. Thus we need to obtain it from:
\begin{framed}
\begin{verbatim}
$ mpicc -show
icc -I/softs/intel//impi/5.0.3.048/intel64/include
-L/softs/intel//impi/5.0.3.048/intel64/lib/release_mt
-L/softs/intel//impi/5.0.3.048/intel64/lib
-Xlinker --enable-new-dtags -Xlinker -rpath
-Xlinker /softs/intel//impi/5.0.3.048/intel64/lib/release_mt
-Xlinker -rpath -Xlinker /softs/intel//impi/5.0.3.048/intel64/lib
-Xlinker -rpath -Xlinker /opt/intel/mpi-rt/5.0/intel64/lib/release_mt
-Xlinker -rpath -Xlinker /opt/intel/mpi-rt/5.0/intel64/lib -lmpifort
-lmpi -lmpigi -ldl -lrt -lpthread
$ 
\end{verbatim}
\end{framed}

From that, we get the MPI configuration section of {\tt project-config.jam}:

\begin{framed}
\begin{verbatim}
...
using python : 2.6 : /usr ; 
#you can insert the section here:
using mpi : mpiicc : 
      <library-path>/softs/intel/impi/5.0.3.048/intel64/lib
      <library-path>/softs/intel/impi/5.0.3.048/intel64/lib/release_mt
      <include>/softs/intel/impi/5.0.3.048/intel64/include
      <find-shared-library>mpifort
      <find-shared-library>mpi_mt
      <find-shared-library>mpigi
      <find-shared-library>dl
      <find-shared-library>rt ;
\end{verbatim}
\end{framed}

Then we build:
\begin{framed}
\begin{verbatim}
$./b2
...
$
\end{verbatim}
\end{framed}
Also, if you are working, for instance, on an 8-core machine you may use \verb+./b2 -j8+  to speed up the process. Finally we install:
\begin{framed}
\begin{verbatim}
$./b2 install
...
$
\end{verbatim}
\end{framed}

\section{Building tt\_inverse3d}

We assume that tt\_inverse3d source files are located in directory \verb+<srcdir>+.
We build the code in directory \verb+<bindir>+ which must be different from \verb+<srcdir>+, usually \verb+<srcdir>+/\verb+<bindir>+.
Provided that the Boost distribution, the MPI distribution and compilers are located in the default places you can build the code with:

\begin{framed}
\begin{verbatim}
$cd <bindir>
$cmake <srcdir>
...
$make [-j]
...
$
\end{verbatim}
\end{framed}

For a successfull compilation ({\bf{cmake}}), you must configure the following variables to agree with your environment:

\begin{description}
\item{\bf BOOST\_ROOT}: the root of your boost installation.
\item{\bf MPI\_COMPILE\_FLAGS}: MPI compilation flags.
\item{\bf MPI\_LINK\_FLAGS}: MPI link flags.
\item{\bf CMAKE\_CXX\_FLAGS}: compiler specific flags. Option {\tt -fp-model precise} is recommended with \intel\ compiler.
\end{description}

If you wish to run the test (recommended), the following extra variables can be useful, depending on the specificities of your MPI distribution:

\begin{description}
\item{\bf MPIEXEC} the MPI launcher, typically {\tt mpiexec} or {\tt mpiexec.hydra}.
\item{\bf MPIEXEC\_NUMPROC\_FLAG} the MPI launcher flag used to specify the number of MPI processes, typically {\tt -n} or {\tt -np}.
\end{description}

The following example is a worst case scenario for a very specific installation:

\begin{framed}
\begin{verbatim}
$ cmake  -DCMAKE_CXX_FLAGS:STRING="-fp-model precise "
-DMPI_COMPILE_FLAGS="-I/softs/intel//impi/5.0.3.048/intel64/include "
-DMPI_LINK_FLAGS="-I/softs/intel//impi/5.0.3.048/intel64/include
-L/softs/intel//impi/5.0.3.048/intel64/lib/release_mt
-L/softs/intel//impi/5.0.3.048/intel64/lib -Xlinker
--enable-new-dtags -Xlinker -rpath -Xlinker /softs/intel//impi/5.0.3.048/intel64/lib/release_mt
-Xlinker -rpath -Xlinker /softs/intel//impi/5.0.3.048/intel64/lib
-Xlinker -rpath -Xlinker /opt/intel/mpi-rt/5.0/intel64/lib/release_mt
-Xlinker -rpath -Xlinker /opt/intel/mpi-rt/5.0/intel64/lib
-lmpifort -lmpi -lmpigi -ldl -lrt -lpthread" -DMPIEXEC=mpiexec.hydra
-DMPIEXEC_NUMPROC_FLAG=-np -DCMAKE_BUILD_TYPE=RelWithDebInfo
-DBOOST_ROOT=/softs/boost-1.59.0-b1-intel15.0.2-impi5.0.3 <srcdir>
...
$
\end{verbatim}
\end{framed}

The script \verb+<srcdir>/preconfigure.sh+ is used by the developers for automatic configuration in their most common platforms. You might want to adapt it to your own.

\section{Testing installation}

After successfully building tt\_inverse3d you can test the installation by running:

\begin{framed}
\begin{verbatim}
$cd <bindir>
$ctest -R bath
...
$
\end{verbatim}
\end{framed}

This will check both MPI and multithreading implementations with a simple synthetic inversion case.
If any of the two tests fails you can check the log file under \verb+<srcdir>+/\verb+<bindir>+/\verb+Testing/Temporary+ for information on the runtime error(s).\\
\newline
A more realistic synthetic inversion test can be found in \verb+<srcdir>+/\verb+applications/Synthetic+.
You can modify and use the inversion script \verb+hybrid.oar+ according to your platform.\\
\newline
We have observed differences in the results when running this test on different machines. These differences are small at the beginning but they sometimes grow with iterations. 
This is caused by the combined effect of different machine and/or compiler floating-point precisions and the iterative exploration of the model space based on previous results: 
a small difference in the first iterations might end up leading to a different (local) minimum. It is important to note that this does not affect the purpose of the code: 
in field data applications, once a final model is obtained on a specific machine, and with specific input data file, initial models, and regularization constraints, 
it must be consistent with common geological knowledge, and ideally its uncertainty and resolution limits should be carefully studied to ensure that its posterior geological interpretation is sound.\\
\newline
In order to check that the code is working properly please look for the following messages in the output file: 
\begin{itemize}
\item \verb+Broadcasting paths from source+ $\#source$. 
\item \verb+TomographicInversion3d::iter=+ $\#current\_iteration(\#total\_iterations)$.
\end{itemize}

\end{document}
